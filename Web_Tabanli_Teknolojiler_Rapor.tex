\documentclass[12pt,a4paper]{article}
\usepackage[utf8]{inputenc}
\usepackage[turkish]{babel}
\usepackage{geometry}
\usepackage{graphicx}
\usepackage{hyperref}
\usepackage{booktabs}
\usepackage{tabularx}
\usepackage{enumitem}
\usepackage{listings}
\usepackage{xcolor}
\usepackage{float}
\usepackage{amsmath}
\usepackage{adjustbox}
\usepackage{array}
\usepackage{longtable}

\geometry{margin=2.5cm}

% Kod renklendirme ayarları
\lstset{
    basicstyle=\ttfamily\small,
    breaklines=true,
    frame=single,
    numbers=left,
    numberstyle=\tiny\color{gray},
    keywordstyle=\color{blue},
    commentstyle=\color{green!60!black},
    stringstyle=\color{red},
    showstringspaces=false
}

\title{\textbf{YEREL HİKAYE PAYLAŞIM PLATFORMU}\\
\large Web Tabanlı Teknolojiler Proje Raporu}
\author{Öğrenci: Alperen CAN\\
Numara: 18253513}
\date{\today}

\begin{document}

\maketitle

\vspace{1cm}
\noindent\textbf{Proje Github Repo:} \url{https://github.com/[github-username]/web_teknolojileri_github.git}

\newpage

\tableofcontents
\newpage

\section{GİRİŞ}

Bu rapor, Yerel Hikaye Paylaşım Platformu için geliştirilmiş olan web uygulamasının teknik detaylarını içermektedir. Proje, modern web teknolojileri kullanılarak tam yığın (full-stack) bir mimari ile geliştirilmiştir. Raporda backend ve frontend bileşenlerinin yapısı, kullanılan teknolojiler ve sistem akışları detaylı olarak açıklanmaktadır.

\section{PROJE ÖZETİ}

Yerel Hikaye Paylaşım Platformu, kullanıcıların yerel hikayelerini paylaşabileceği, harita üzerinde konumlandırabileceği ve diğer kullanıcılarla etkileşim kurabileceği kapsamlı bir web uygulamasıdır. Sistem, kullanıcıların hikaye oluşturmasını, düzenlemesini ve paylaşmasını sağlarken, yöneticilerin içerik ve kullanıcı yönetimi yapmasına olanak tanımaktadır.

\subsection{Projenin Temel Özellikleri}

\begin{itemize}
    \item Kullanıcı kayıt ve giriş sistemi
    \item Rol tabanlı yetkilendirme (User, Admin)
    \item Hikaye oluşturma, düzenleme ve paylaşma
    \item Kategori yönetimi
    \item Yorum ve beğeni sistemi
    \item Google Maps entegrasyonu ile konum bazlı hikaye paylaşımı
    \item Fotoğraf yükleme ve galeri görüntüleme
    \item Responsive (duyarlı) tasarım
    \item Tek sayfa uygulama (SPA) mimarisi
\end{itemize}

\section{TEKNOLOJİ YIĞINI (TECHNOLOGY STACK)}

\subsection{Backend Teknolojileri}

\begin{table}[H]
\centering
\adjustbox{width=\textwidth,center}
{
\begin{tabularx}{\textwidth}{|p{4cm}|X|}
\hline
\textbf{Teknoloji} & \textbf{Açıklama} \\
\hline
NestJS & Node.js tabanlı sunucu framework'ü, modüler mimari ve TypeScript desteği \\
\hline
TypeORM & Object-Relational Mapping aracı, veritabanı işlemlerini kolaylaştırır \\
\hline
PostgreSQL & İlişkisel veritabanı yönetim sistemi, güvenilir ve ölçeklenebilir \\
\hline
JWT (JSON Web Token) & Kimlik doğrulama mekanizması, stateless authentication \\
\hline
Swagger & API dokümantasyonu, interaktif API test arayüzü \\
\hline
bcrypt & Parola şifreleme, güvenli hash algoritması \\
\hline
Multer & Dosya yükleme middleware'i, fotoğraf yükleme işlemleri \\
\hline
Passport & Kimlik doğrulama middleware'i, JWT stratejisi ile entegrasyon \\
\hline
class-validator & Veri doğrulama, DTO'lar için validation decorator'ları \\
\hline
class-transformer & Veri dönüşümü, nesne mapping işlemleri \\
\hline
\end{tabularx}
}
\caption{Backend Teknolojileri}
\end{table}

\subsection{Frontend Teknolojileri}

\begin{table}[H]
\centering
\adjustbox{width=\textwidth,center}
{
\begin{tabularx}{\textwidth}{|p{4cm}|X|}
\hline
\textbf{Teknoloji} & \textbf{Açıklama} \\
\hline
React 18 & Kullanıcı arayüzü kütüphanesi, component tabanlı mimari \\
\hline
TypeScript & Tip güvenli JavaScript süper seti, geliştirme hatalarını önler \\
\hline
React Router & Client-side yönlendirme, SPA routing yönetimi \\
\hline
Axios & HTTP istemci kütüphanesi, RESTful API çağrıları \\
\hline
Vite & Modern build aracı, hızlı geliştirme ve build süreçleri \\
\hline
Tailwind CSS & Utility-first CSS framework, hızlı ve responsive tasarım \\
\hline
React Hot Toast & Bildirim kütüphanesi, kullanıcı geri bildirimleri \\
\hline
@react-google-maps/api & Google Maps React entegrasyonu, harita görüntüleme \\
\hline
React Leaflet & Leaflet harita kütüphanesi, alternatif harita görüntüleme \\
\hline
Framer Motion & Animasyon kütüphanesi, akıcı UI animasyonları \\
\hline
Lucide React & İkon kütüphanesi, modern ve tutarlı ikonlar \\
\hline
\end{tabularx}
}
\caption{Frontend Teknolojileri}
\end{table}

\section{VERİTABANI MİMARİSİ}

\subsection{Varlık İlişki Modeli (Entity Relationship)}

Sistem, aşağıdaki ilişkileri içeren bir veritabanı yapısına sahiptir:

\begin{itemize}
    \item \textbf{USERS (1) $\rightarrow$ PROFILE (1)}: Bir kullanıcının bir profili olabilir (One-to-One)
    \item \textbf{USERS (1) $\rightarrow$ STORIES (N)}: Bir kullanıcının birden fazla hikayesi olabilir (One-to-Many)
    \item \textbf{USERS (1) $\rightarrow$ COMMENTS (N)}: Bir kullanıcının birden fazla yorumu olabilir (One-to-Many)
    \item \textbf{STORIES (N) $\leftrightarrow$ CATEGORIES (N)}: Bir hikaye birden fazla kategoride olabilir, bir kategoride birden fazla hikaye olabilir (Many-to-Many)
    \item \textbf{STORIES (1) $\rightarrow$ COMMENTS (N)}: Bir hikayeye birden fazla yorum yapılabilir (One-to-Many)
    \item \textbf{STORIES (1) $\rightarrow$ STORY\_LIKES (N)}: Bir hikayeye birden fazla beğeni yapılabilir (One-to-Many)
    \item \textbf{COMMENTS (1) $\rightarrow$ COMMENT\_LIKES (N)}: Bir yoruma birden fazla beğeni yapılabilir (One-to-Many)
\end{itemize}

\subsection{ENUM Değerleri}

\begin{itemize}
    \item \textbf{UserRole}: User | Admin
    \item \textbf{LikeAction}: like | dislike
\end{itemize}

\subsection{Veritabanı Tabloları}

\subsubsection{USERS Tablosu}
\begin{table}[H]
\centering
\adjustbox{width=\textwidth,center}
{
\begin{tabularx}{\textwidth}{|p{3cm}|p{4cm}|p{8cm}|}
\hline
\textbf{Alan} & \textbf{Tip} & \textbf{Açıklama} \\
\hline
id & UUID (PK) & Benzersiz kullanıcı kimliği \\
\hline
email & VARCHAR (UNIQUE) & Kullanıcı e-posta adresi \\
\hline
username & VARCHAR & Kullanıcı adı \\
\hline
password & VARCHAR & Şifrelenmiş parola (bcrypt) \\
\hline
role & VARCHAR(20) & Kullanıcı rolü (User/Admin) \\
\hline
isActive & BOOLEAN & Kullanıcı aktiflik durumu \\
\hline
createdAt & TIMESTAMP & Oluşturulma tarihi \\
\hline
updatedAt & TIMESTAMP & Güncellenme tarihi \\
\hline
\end{tabularx}
}
\caption{USERS Tablosu}
\end{table}

\subsubsection{PROFILES Tablosu}
\begin{table}[H]
\centering
\adjustbox{width=\textwidth,center}
{
\begin{tabularx}{\textwidth}{|p{3cm}|p{4cm}|p{8cm}|}
\hline
\textbf{Alan} & \textbf{Tip} & \textbf{Açıklama} \\
\hline
id & UUID (PK) & Benzersiz profil kimliği \\
\hline
userId & UUID (FK) & Kullanıcı kimliği (USERS tablosuna referans) \\
\hline
firstName & VARCHAR & Ad \\
\hline
lastName & VARCHAR & Soyad \\
\hline
bio & TEXT & Biyografi \\
\hline
avatar & VARCHAR & Profil fotoğrafı URL'i \\
\hline
phone & VARCHAR & Telefon numarası \\
\hline
location & VARCHAR & Konum bilgisi \\
\hline
createdAt & TIMESTAMP & Oluşturulma tarihi \\
\hline
updatedAt & TIMESTAMP & Güncellenme tarihi \\
\hline
\end{tabularx}
}
\caption{PROFILES Tablosu}
\end{table}

\subsubsection{STORIES Tablosu}
\begin{table}[H]
\centering
\adjustbox{width=\textwidth,center}
{
\begin{tabularx}{\textwidth}{|p{3cm}|p{4cm}|p{8cm}|}
\hline
\textbf{Alan} & \textbf{Tip} & \textbf{Açıklama} \\
\hline
id & UUID (PK) & Benzersiz hikaye kimliği \\
\hline
authorId & UUID (FK) & Yazar kimliği (USERS tablosuna referans) \\
\hline
title & VARCHAR & Hikaye başlığı \\
\hline
content & TEXT & Hikaye içeriği \\
\hline
latitude & DECIMAL(10,8) & Enlem koordinatı (Google Maps) \\
\hline
longitude & DECIMAL(11,8) & Boylam koordinatı (Google Maps) \\
\hline
locationName & VARCHAR & Lokasyon adı \\
\hline
photos & SIMPLE-ARRAY & Fotoğraf URL'leri dizisi \\
\hline
likes & INTEGER & Beğeni sayısı \\
\hline
dislikes & INTEGER & Beğenmeme sayısı \\
\hline
isPublished & BOOLEAN & Yayınlanma durumu \\
\hline
createdAt & TIMESTAMP & Oluşturulma tarihi \\
\hline
updatedAt & TIMESTAMP & Güncellenme tarihi \\
\hline
\end{tabularx}
}
\caption{STORIES Tablosu}
\end{table}

\subsubsection{CATEGORIES Tablosu}
\begin{table}[H]
\centering
\adjustbox{width=\textwidth,center}
{
\begin{tabularx}{\textwidth}{|p{3cm}|p{4cm}|p{8cm}|}
\hline
\textbf{Alan} & \textbf{Tip} & \textbf{Açıklama} \\
\hline
id & UUID (PK) & Benzersiz kategori kimliği \\
\hline
name & VARCHAR & Kategori adı \\
\hline
description & TEXT & Kategori açıklaması \\
\hline
icon & VARCHAR & Kategori ikonu \\
\hline
createdAt & TIMESTAMP & Oluşturulma tarihi \\
\hline
updatedAt & TIMESTAMP & Güncellenme tarihi \\
\hline
\end{tabularx}
}
\caption{CATEGORIES Tablosu}
\end{table}

\subsubsection{COMMENTS Tablosu}
\begin{table}[H]
\centering
\adjustbox{width=\textwidth,center}
{
\begin{tabularx}{\textwidth}{|p{3cm}|p{4cm}|p{8cm}|}
\hline
\textbf{Alan} & \textbf{Tip} & \textbf{Açıklama} \\
\hline
id & UUID (PK) & Benzersiz yorum kimliği \\
\hline
authorId & UUID (FK) & Yazar kimliği (USERS tablosuna referans) \\
\hline
storyId & UUID (FK) & Hikaye kimliği (STORIES tablosuna referans) \\
\hline
content & TEXT & Yorum içeriği \\
\hline
likes & INTEGER & Beğeni sayısı \\
\hline
dislikes & INTEGER & Beğenmeme sayısı \\
\hline
createdAt & TIMESTAMP & Oluşturulma tarihi \\
\hline
updatedAt & TIMESTAMP & Güncellenme tarihi \\
\hline
\end{tabularx}
}
\caption{COMMENTS Tablosu}
\end{table}

\subsubsection{STORY\_CATEGORIES Tablosu (Join Table)}
\begin{table}[H]
\centering
\adjustbox{width=\textwidth,center}
{
\begin{tabularx}{\textwidth}{|p{5cm}|p{5cm}|p{5cm}|}
\hline
\textbf{Alan} & \textbf{Tip} & \textbf{Açıklama} \\
\hline
storyId & UUID (FK) & Hikaye kimliği \\
\hline
categoryId & UUID (FK) & Kategori kimliği \\
\hline
\end{tabularx}
}
\caption{STORY\_CATEGORIES Tablosu (Many-to-Many Join)}
\end{table}

\subsubsection{STORY\_LIKES Tablosu}
\begin{table}[H]
\centering
\adjustbox{width=\textwidth,center}
{
\begin{tabularx}{\textwidth}{|p{3cm}|p{4cm}|p{8cm}|}
\hline
\textbf{Alan} & \textbf{Tip} & \textbf{Açıklama} \\
\hline
id & UUID (PK) & Benzersiz beğeni kimliği \\
\hline
userId & UUID (FK) & Kullanıcı kimliği \\
\hline
storyId & UUID (FK) & Hikaye kimliği \\
\hline
action & VARCHAR & Beğeni türü (like/dislike) \\
\hline
createdAt & TIMESTAMP & Oluşturulma tarihi \\
\hline
\end{tabularx}
}
\caption{STORY\_LIKES Tablosu}
\end{table}

\subsubsection{COMMENT\_LIKES Tablosu}
\begin{table}[H]
\centering
\adjustbox{width=\textwidth,center}
{
\begin{tabularx}{\textwidth}{|p{3cm}|p{4cm}|p{8cm}|}
\hline
\textbf{Alan} & \textbf{Tip} & \textbf{Açıklama} \\
\hline
id & UUID (PK) & Benzersiz beğeni kimliği \\
\hline
userId & UUID (FK) & Kullanıcı kimliği \\
\hline
commentId & UUID (FK) & Yorum kimliği \\
\hline
action & VARCHAR & Beğeni türü (like/dislike) \\
\hline
createdAt & TIMESTAMP & Oluşturulma tarihi \\
\hline
\end{tabularx}
}
\caption{COMMENT\_LIKES Tablosu}
\end{table}

\section{BACKEND API ENDPOINT'LERİ}

Backend, RESTful API prensiplerine uygun olarak tasarlanmıştır. Aşağıda tüm endpoint'ler modüllere göre kategorize edilmiş şekilde açıklanmaktadır.

\subsection{Kimlik Doğrulama Modülü (Auth Module)}

\begin{table}[H]
\centering
\adjustbox{width=\textwidth,center}
{
\begin{tabularx}{\textwidth}{|p{2cm}|p{4cm}|p{3cm}|X|}
\hline
\textbf{Metod} & \textbf{Endpoint} & \textbf{Erişim} & \textbf{Açıklama} \\
\hline
POST & /auth/register & Herkese Açık & Yeni kullanıcı kaydı oluşturur. E-posta, kullanıcı adı, parola ve profil bilgilerini alır, başarılı kayıt sonrası JWT token döndürür. \\
\hline
POST & /auth/login & Herkese Açık & Mevcut kullanıcı girişi yapar. E-posta ve parola doğrulaması yaparak JWT token döndürür. \\
\hline
GET & /auth/me & Giriş Yapmış & Mevcut oturumdaki kullanıcının bilgilerini döndürür. Token doğrulaması için kullanılır. \\
\hline
\end{tabularx}
}
\caption{Auth Modülü Endpoint'leri}
\end{table}

\subsection{Kullanıcı Modülü (Users Module)}

\begin{table}[H]
\centering
\adjustbox{width=\textwidth,center}
{
\begin{tabularx}{\textwidth}{|p{2cm}|p{4cm}|p{3cm}|X|}
\hline
\textbf{Metod} & \textbf{Endpoint} & \textbf{Erişim} & \textbf{Açıklama} \\
\hline
GET & /users & Admin & Tüm kullanıcıları listeler. \\
\hline
GET & /users/me & Giriş Yapmış & Mevcut kullanıcının bilgilerini getirir. \\
\hline
GET & /users/:id & Giriş Yapmış & Belirli bir kullanıcının detaylarını getirir. \\
\hline
PATCH & /users/:id & Giriş Yapmış & Kullanıcı bilgilerini günceller. Sadece kendi profilini veya Admin tüm profilleri güncelleyebilir. \\
\hline
DELETE & /users/:id & Admin & Kullanıcıyı siler. \\
\hline
\end{tabularx}
}
\caption{Users Modülü Endpoint'leri}
\end{table}

\subsection{Hikaye Modülü (Stories Module)}

\begin{table}[H]
\centering
\adjustbox{width=\textwidth,center}
{
\begin{tabularx}{\textwidth}{|p{2cm}|p{4cm}|p{3cm}|X|}
\hline
\textbf{Metod} & \textbf{Endpoint} & \textbf{Erişim} & \textbf{Açıklama} \\
\hline
GET & /stories & Herkese Açık & Tüm hikayeleri listeler. Varsayılan olarak sadece yayınlanmış hikayeleri gösterir. Query parametresi ile tüm hikayeler görüntülenebilir. \\
\hline
GET & /stories/my & Giriş Yapmış & Mevcut kullanıcının hikayelerini listeler. \\
\hline
GET & /stories/:id & Herkese Açık & Belirli bir hikayenin detaylarını getirir. İlişkili verilerle birlikte (yazar, kategoriler, yorumlar). \\
\hline
POST & /stories & Giriş Yapmış & Yeni hikaye oluşturur. Başlık, içerik, koordinatlar, kategoriler ve fotoğraflar alınır. \\
\hline
PATCH & /stories/:id & Giriş Yapmış & Hikaye bilgilerini günceller. Sadece hikaye sahibi veya Admin güncelleyebilir. \\
\hline
DELETE & /stories/:id & Giriş Yapmış & Hikayeyi siler. Sadece hikaye sahibi veya Admin silebilir. \\
\hline
POST & /stories/:id/like & Giriş Yapmış & Hikayeyi beğenir veya beğenmez. Her kullanıcı bir kez tepki verebilir, aynı tepkiyi tekrar verirse geri alınır. \\
\hline
GET & /stories/:id/like-status & Giriş Yapmış & Kullanıcının bu hikayeye verdiği tepkiyi getirir (like/dislike/null). \\
\hline
\end{tabularx}
}
\caption{Stories Modülü Endpoint'leri}
\end{table}

\subsection{Kategori Modülü (Categories Module)}

\begin{table}[H]
\centering
\adjustbox{width=\textwidth,center}
{
\begin{tabularx}{\textwidth}{|p{2cm}|p{4cm}|p{3cm}|X|}
\hline
\textbf{Metod} & \textbf{Endpoint} & \textbf{Erişim} & \textbf{Açıklama} \\
\hline
GET & /categories & Herkese Açık & Tüm kategorileri listeler. \\
\hline
GET & /categories/:id & Herkese Açık & Belirli bir kategorinin detaylarını getirir. \\
\hline
POST & /categories & Admin & Yeni kategori oluşturur. Kategori adı ve açıklaması alınır. \\
\hline
PATCH & /categories/:id & Admin & Kategori bilgilerini günceller. \\
\hline
DELETE & /categories/:id & Admin & Kategoriyi siler. \\
\hline
\end{tabularx}
}
\caption{Categories Modülü Endpoint'leri}
\end{table}

\subsection{Yorum Modülü (Comments Module)}

\begin{table}[H]
\centering
\adjustbox{width=\textwidth,center}
{
\begin{tabularx}{\textwidth}{|p{2cm}|p{4cm}|p{3cm}|X|}
\hline
\textbf{Metod} & \textbf{Endpoint} & \textbf{Erişim} & \textbf{Açıklama} \\
\hline
GET & /comments & Herkese Açık & Tüm yorumları listeler. Query parametresi ile belirli bir hikayeye ait yorumlar filtrelenebilir. \\
\hline
GET & /comments/:id & Herkese Açık & Belirli bir yorumun detaylarını getirir. \\
\hline
POST & /comments & Giriş Yapmış & Yeni yorum oluşturur. Yorum içeriği ve hikaye ID'si alınır. \\
\hline
PATCH & /comments/:id & Giriş Yapmış & Yorum bilgilerini günceller. Sadece yorum sahibi veya Admin güncelleyebilir. \\
\hline
DELETE & /comments/:id & Giriş Yapmış & Yorumu siler. Sadece yorum sahibi veya Admin silebilir. \\
\hline
POST & /comments/:id/like & Giriş Yapmış & Yorumu beğenir veya beğenmez. Her kullanıcı bir kez tepki verebilir. \\
\hline
GET & /comments/:id/like-status & Giriş Yapmış & Kullanıcının bu yoruma verdiği tepkiyi getirir. \\
\hline
\end{tabularx}
}
\caption{Comments Modülü Endpoint'leri}
\end{table}

\subsection{Yükleme Modülü (Upload Module)}

\begin{table}[H]
\centering
\adjustbox{width=\textwidth,center}
{
\begin{tabularx}{\textwidth}{|p{2cm}|p{4cm}|p{3cm}|X|}
\hline
\textbf{Metod} & \textbf{Endpoint} & \textbf{Erişim} & \textbf{Açıklama} \\
\hline
POST & /upload/photo & Giriş Yapmış & Fotoğraf yükler. Multipart/form-data formatında dosya alınır. Sadece resim dosyalarına izin verilir (jpg, jpeg, png, gif). Maksimum dosya boyutu 5MB'dır. Yüklenen dosya /uploads klasörüne kaydedilir ve URL döndürülür. \\
\hline
\end{tabularx}
}
\caption{Upload Modülü Endpoint'leri}
\end{table}

\section{FRONTEND SAYFALARI VE ROUTING}

Frontend, React Router kullanılarak tek sayfa uygulama (SPA) mimarisi ile geliştirilmiştir. Aşağıda tüm sayfalar ve routing yapılandırması açıklanmaktadır.

\subsection{Routing Yapılandırması}

\begin{table}[H]
\centering
\adjustbox{width=\textwidth,center}
{
\begin{tabularx}{\textwidth}{|p{3cm}|p{3cm}|p{3cm}|X|}
\hline
\textbf{Route} & \textbf{Bileşen} & \textbf{Erişim} & \textbf{Açıklama} \\
\hline
/ & Home & Herkese Açık & Ana sayfa, platform tanıtımı ve hızlı erişim linkleri \\
\hline
/login & Login & Herkese Açık & Kullanıcı giriş sayfası \\
\hline
/register & Register & Herkese Açık & Kullanıcı kayıt sayfası \\
\hline
/stories & Stories & Herkese Açık & Tüm hikayelerin listelendiği sayfa, liste/harita görünümü \\
\hline
/stories/:id & StoryDetail & Herkese Açık & Hikaye detay sayfası, içerik, fotoğraflar, yorumlar ve harita \\
\hline
/my-stories & MyStories & Giriş Yapmış & Kullanıcının kendi hikayelerini listelediği sayfa \\
\hline
/create-story & CreateStory & Giriş Yapmış & Yeni hikaye oluşturma sayfası \\
\hline
/edit-story/:id & EditStory & Giriş Yapmış & Hikaye düzenleme sayfası \\
\hline
/profile & Profile & Giriş Yapmış & Kullanıcı profil sayfası, ayarlar ve etkileşimler \\
\hline
/admin & AdminPanel & Admin & Yönetim paneli, kullanıcı, kategori ve hikaye yönetimi \\
\hline
\end{tabularx}
}
\caption{Frontend Routing Yapılandırması}
\end{table}

\subsection{Korumalı Route'lar}

Sistem, iki tip korumalı route kullanmaktadır:

\begin{itemize}
    \item \textbf{ProtectedRoute}: Giriş yapmış kullanıcılar için. Giriş yapmamış kullanıcılar login sayfasına yönlendirilir.
    \item \textbf{AdminRoute}: Admin rolü için. Yetkisiz kullanıcılar ana sayfaya yönlendirilir.
\end{itemize}

\subsection{Ana Bileşenler}

\subsubsection{Home Sayfası}
Ana sayfa, platformun tanıtımını yapar ve kullanıcılara hızlı erişim linkleri sunar. Hero section, özellik kartları ve call-to-action bölümleri içerir.

\subsubsection{Stories Sayfası}
Hikayelerin listelendiği sayfa. Liste ve harita görünümü arasında geçiş yapılabilir. Kategori filtresi ve arama özelliği bulunur. Her hikaye kartında beğeni/beğenmeme butonları yer alır.

\subsubsection{StoryDetail Sayfası}
Hikaye detay sayfası, hikaye içeriğini, fotoğraf galerisini, yorumları ve harita görünümünü gösterir. Kullanıcılar yorum yapabilir ve beğeni/beğenmeme işlemi gerçekleştirebilir.

\subsubsection{CreateStory/EditStory Sayfaları}
Hikaye oluşturma ve düzenleme formları. Başlık, içerik, lokasyon seçimi (Google Maps entegrasyonu), kategori seçimi ve fotoğraf yükleme özellikleri içerir.

\subsubsection{AdminPanel Sayfası}
Yönetim paneli, Admin kullanıcıları için kullanıcı, kategori ve hikaye yönetimi sekmeleri içerir. CRUD işlemleri gerçekleştirilebilir.

\subsubsection{Profile Sayfası}
Kullanıcı profil sayfası, ayarlar ve etkileşimler sekmeleri içerir. Kullanıcı bilgileri güncellenebilir, profil fotoğrafı yüklenebilir ve kullanıcının hikayeleri ile yorumları görüntülenebilir.

\section{BAĞLAM YÖNETİMİ (CONTEXT MANAGEMENT)}

\subsection{AuthContext}

Uygulamanın merkezi kimlik doğrulama durumunu yönetir. Kullanıcı bilgileri, giriş, kayıt ve çıkış fonksiyonlarını tüm bileşenlere React Context API aracılığıyla sağlar.

\subsubsection{State Yönetimi}
\begin{itemize}
    \item \textbf{user}: Mevcut kullanıcı bilgileri
    \item \textbf{token}: JWT authentication token
    \item \textbf{isAuthenticated}: Kullanıcının giriş yapıp yapmadığı bilgisi
    \item \textbf{isAdmin}: Kullanıcının Admin rolüne sahip olup olmadığı
\end{itemize}

\subsubsection{Fonksiyonlar}
\begin{itemize}
    \item \textbf{login(email, password)}: Kullanıcı girişi yapar, token'ı localStorage'a kaydeder
    \item \textbf{register(userData)}: Yeni kullanıcı kaydı oluşturur, token'ı localStorage'a kaydeder
    \item \textbf{logout()}: Kullanıcı çıkışı yapar, token'ı temizler
    \item \textbf{checkAuth()}: Token geçerliliğini kontrol eder, uygulama başlatılırken çağrılır
\end{itemize}

\section{GÜVENLİK ÖZELLİKLERİ}

\subsection{Kimlik Doğrulama ve Yetkilendirme}
\begin{itemize}
    \item JWT token tabanlı stateless authentication
    \item Parolalar bcrypt ile hash'lenerek saklanır
    \item Token'lar HTTP-only cookie veya localStorage'da saklanır
    \item Token geçerliliği her istekte kontrol edilir
\end{itemize}

\subsection{Rol Tabanlı Erişim Kontrolü}
\begin{itemize}
    \item User: Normal kullanıcı, hikaye oluşturma ve yorum yapma yetkisi
    \item Admin: Tüm yetkilere sahip, kullanıcı, kategori ve içerik yönetimi yetkisi
\end{itemize}

\subsection{Veri Doğrulama}
\begin{itemize}
    \item Backend'de class-validator ile DTO doğrulama
    \item Frontend'de form validation
    \item SQL injection koruması (TypeORM parametreli sorgular)
    \item XSS koruması (React otomatik escaping)
\end{itemize}

\section{SONUÇ}

Yerel Hikaye Paylaşım Platformu, modern web geliştirme prensipleri ve endüstri standartları gözetilerek geliştirilmiş kapsamlı bir tam yığın uygulamadır. NestJS ile oluşturulan backend, RESTful API standartlarına uygun, güvenli ve ölçeklenebilir bir yapı sunarken; React ile geliştirilen frontend, kullanıcı dostu ve responsive bir arayüz sağlamaktadır.

Sistemin modüler mimarisi, gelecekteki geliştirmeler ve bakım işlemleri için uygun bir zemin oluşturmaktadır. Rol tabanlı yetkilendirme, veri doğrulama ve şifreli parola saklama gibi güvenlik önlemleri, uygulamanın güvenilirliğini artırmaktadır. Google Maps entegrasyonu ile konum bazlı hikaye paylaşımı, kullanıcı deneyimini zenginleştirmektedir.

Proje, SOLID prensipleri ve Clean Architecture yaklaşımı ile geliştirilmiş, kod tekrarını önleyen ve bakımı kolay bir yapıya sahiptir. Swagger ile dokümante edilmiş API'ler, geliştiriciler için kolay entegrasyon imkanı sunmaktadır.

\section{EK BİLGİLER}

\subsection{Kurulum ve Çalıştırma}

Backend ve frontend ayrı dizinlerde bulunmakta ve bağımsız olarak çalıştırılabilmektedir. Her iki uygulama da kendi bağımlılıklarını yönetir ve geliştirme ortamında hot-reload desteği sunar.

\subsection{API Dokümantasyonu}

Tüm API endpoint'leri Swagger UI ile dokümante edilmiştir. Backend çalıştırıldığında \texttt{http://localhost:3000/api} adresinden erişilebilir.

\subsection{Geliştirme Notları}

\begin{itemize}
    \item Kod yapısı modüler ve temiz tutulmuştur
    \item SOLID prensipleri ve Clean Architecture yaklaşımı uygulanmıştır
    \item TypeScript kullanılarak tip güvenliği sağlanmıştır
\end{itemize}

\end{document}
